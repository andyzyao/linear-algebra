% introduction-to-linear-algebra.tex
% xelatex ./introduction-to-linear-algebra.tex 
\documentclass[UTF8]{ctexart}

\title{Introduction to Linear Algebra学习笔记}
\author{Andy Yao}
\date{\today}

\begin{document}

\maketitle

\section{Introduction to Vectors}
  \subsection{Vectors and Linear Combinations}
  \subsection{Lengths and Dot Porducts}
  \subsection{Matrices}

\section{Solving Linear Equations}
  \subsection{Vectors and Linear Equations}
  \subsection{The Idea of Elimination}
  \subsection{Elimination Using Matrices}
1. $Ax = x_{1}$ times column 1 + ... + $x_{n}$ times column n. And $(Ax)_{i} = \sum_{j=1}^n a_{ij}x_{j}$.

2. Identity matrix $= I$, elimination matrix = $E_{ij}$ using $l_{ij}$, exchange matrix $= P_{ij}$.

3. Multiplying Ax = b by E21 subtracts a multiple l21 of equation 1 from equation 2. The number -l21 is the (2.1) entry
of the elimination matrix E21.

4. For the augmented matrix [A b], that elimination step gives [E21A E21b].

5. When A multiplies any matrix B, it multiplies each column of B separately.

  \subsection{Rules for Matrix Operations}
  \subsection{Inverse Matrices}
  \subsection{Elimination = Factorization: A = LU}
  \subsection{Transposes and Permutations}

\end{document}


